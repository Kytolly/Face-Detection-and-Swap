\\section{实验步骤}\n\n本实验按照以下步骤实现并评估了不同的场景分类方法:\n\n\\subsection{数据准备}\n\n首先,加载场景图像数据集,并将其划分为训练集和测试集。数据集路径和划分比例在代码中预先设定。\n\n\\subsection{基于 Tiny Images 和 KNN 的场景分类}\n\n\\begin{enumerate}\n    \\item \\textbf{特征提取:} 对训练集和测试集中的每张图像,执行以下操作生成 Tiny Images 特征:\n    \\begin{itemize}\n        \\item 读取图像并转换为灰度。\n        \\item 将图像缩放到固定的小尺寸(例如 $16 \\times 16$ 像素)。\n        \\item 将缩放后的图像像素值展平为一维特征向量。\n        \\item 对特征向量进行归一化(例如,减去均值并除以标准差)。\n    \\end{itemize}\n    得到训练集和测试集的 Tiny Images 特征矩阵。\n    \\item \\textbf{分类:} 使用 K 近邻 (KNN) 分类器。对于测试集中的每个样本,计算其与所有训练集样本的欧氏距离,找出最近的 $K$ 个邻居(实验中 $K$ 的值可调,例如取 $K=1$ 或其他值),然后根据这 $K$ 个邻居的多数投票决定测试样本的类别。\n    \\item \\textbf{评估:} 计算 KNN 分类器在测试集上的分类准确率。\n\\end{enumerate}\n\n\\subsection{基于 Bag of Words (SIFT) 和 SVM 的场景分类}\n\n\\begin{enumerate}\n    \\item \\textbf{构建视觉词汇表:}\n    \\begin{itemize}\n        \\item 从训练集图像中提取大量的 SIFT 局部特征描述符。\n        \\item 使用 MiniBatchKMeans 聚类算法对收集到的 SIFT 特征进行聚类,聚类中心数量设定为词汇表大小 (vocab\_size)。\n        \\item 将聚类中心保存为视觉词汇表。\n    \\end{itemize}\n    \\item \\textbf{生成 Bag of Words 特征:}\n    \\begin{itemize}\n        \\item 对训练集和测试集中的每张图像,提取其 SIFT 局部特征。\n        \\item 对于每个提取到的 SIFT 特征,在视觉词汇表中找到距离最近的视觉词汇。\n        \\item 统计每个视觉词汇在该图像中出现的频率,生成该图像的 Bag of Words 直方图特征向量。\n    \\end{itemize}\n    得到训练集和测试集的 Bag of Words 特征矩阵。\n    \\item \\textbf{分类:} 使用支持向量机 (SVM) 分类器(具体为 LinearSVC)。使用训练集的 Bag of Words 特征和标签训练 SVM 分类器(采用 One-vs-Rest 策略处理多类别问题)。\n    \\item \\textbf{评估:} 使用训练好的 SVM 分类器预测测试集图像的类别,并计算分类准确率。\n\\end{enumerate}\n\n\\subsection{基于深度神经网络 (DNN) 的场景分类(如果实现)}\n\n\\begin{enumerate}\n    \\item \\textbf{模型构建:} 定义一个卷积神经网络 (CNN) 模型架构,包括卷积层、激活层、池化层和全连接层等。\n    \\item \\textbf{数据处理:} 对图像数据进行必要的预处理和数据增强,并使用数据加载器进行批量处理。\n    \\item \\textbf{模型训练:} 定义损失函数(如交叉熵)和优化器。使用训练集数据训练 DNN 模型,通过反向传播更新模型参数。\n    \\item \\textbf{评估:} 在测试集上评估训练好的 DNN 模型的分类性能,计算准确率。\n\\end{enumerate}\n\n\\subsection{结果比较与分析}\n\n最后,比较不同方法(Tiny Images + KNN, Bag of Words + SVM, 以及 DNN)在测试集上的分类准确率。分析不同特征表示方法和分类器对场景分类性能的影响,讨论各方法的优劣和适用场景。 