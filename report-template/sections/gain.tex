通过本次面部识别与换脸的实验,我获得了以下心得体会:

我通过亲手实现面部检测、关键点定位、Delaunay三角剖分、仿射变换和图像融合等步骤,
让我对这些图像处理的核心概念有了更深刻的理解。
特别是Delaunay三角剖分在保持面部结构完整性方面的巧妙应用,
以及泊松融合在实现无缝过渡中的重要性,都让我印象深刻,让我深入了理解图像处理核心概念。


在实验过程中,我大量使用了OpenCV和Dlib库的各种函数。这极大地提升了我对这两个库的熟练程度和实际应用能力。
从零开始构建一个相对复杂的图像处理应用,让我学会了如何将一个大问题分解为多个小模块,并逐一攻克。
在调试过程中,也锻炼了我的问题定位和解决能力。

通过手写实现与大模型Demo的对比,我直观地感受到了传统图像处理方法在特定场景下的局限性,
以及深度学习模型在处理复杂图像任务(如面部表情、光照变化)时所展现出的强大能力和更高的自然度。
这促使我思考未来在图像处理领域,如何更好地结合传统算法和深度学习技术。
    
总而言之,本次实验不仅成功实现了面部换脸功能,更重要的是,
它为我提供了一个将理论知识应用于实践的宝贵机会,
加深了我对计算机视觉和图像处理领域的理解,并激发了我对未来进一步探索的兴趣。