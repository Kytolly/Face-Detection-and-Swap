尽管本次实验成功实现了面部识别与换脸的基本功能,
但在实际应用中仍有许多可以改进的方向,
以提升其鲁棒性、自然度和用户体验:

\begin{enumerate}
    \item 现有方法在处理面部姿态(如侧脸、低头)或夸张表情差异较大的图像时,换脸效果可能不佳。
    可以引入更复杂的3D面部模型或姿态归一化技术,将人脸对齐到标准姿态,以提高换脸的鲁棒性。

    \item 融合后的面部可能与目标图像的原始光照条件和肤色不完全匹配,导致不自然。
    可以引入更高级的颜色校正和光照估计技术,如直方图匹配、颜色迁移算法或基于深度学习的光照重建,使融合更加自然。

    \item 将当前针对单张图像的换脸技术扩展到视频序列。这需要考虑帧与帧之间的一致性、时间平滑性以及实时处理能力。
\end{enumerate}
