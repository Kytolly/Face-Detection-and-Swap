\subsection{手写面部识别}
本实验基于OpenCV和Dlib库实现面部检测和换脸,主要包含以下步骤:

\begin{enumerate}
    \item \textbf{环境配置和导入} 
	安装实验所需的库,特别是Dlib库,这通常需要预先安装Visual Studio和CMake。
	导入必要的Python库,例如 \texttt{cv2}, \texttt{numpy}, \texttt{dlib}, \texttt{PIL}, \texttt{matplotlib.pyplot} 
	读取事先准备用于换脸的两张人物图像,并将它们的大小统一调整为 (300, 300) 像素。

    \item \textbf{面部检测和关键点定位:} 
	利用Dlib的 \texttt{get_frontal_face_detector()} 在图像中定位人脸区域,
	并使用预训练的 \texttt{shape_predictor} 检测出人脸的68个关键点。
	此步骤通过实现 \texttt{get_landmarks} 函数完成,该函数负责检测人脸并返回关键点坐标。
	同时,实现 \texttt{get_face_mask} 函数,通过计算面部关键点的凸包来生成面部遮罩,确保后续操作只在面部区域进行。

    \item \textbf{面部网格化:} 
	以检测到的面部关键点作为顶点,
	对源图像和目标图像的面部区域进行Delaunay三角剖分,构建面部网格。
	此步骤通过实现 \texttt{get_delaunay_triangulation} 函数完成,它基于关键点生成面部的三角形网格。

    \item \textbf{面部对齐:} 
	根据源图像和目标图像的关键点,计算仿射变换矩阵,对源图像的面部进行对齐。
	这通过实现 \texttt{transformation_from_landmarks} 函数完成,
	该函数计算一个部分仿射变换矩阵,将源面部关键点映射到目标面部关键点。

    \item \textbf{变形:} 
	对齐后的源面部图像根据目标面部的网格进行仿射变形,使其形状与目标面部更匹配。
	这通过实现 \texttt{warp_img} 函数完成,该函数使用仿射变换矩阵对图像进行几何变换。
    
	\item \textbf{无缝融合:} 
	将变形后的源面部区域融合到目标图像中。
	首先,遍历Delaunay三角剖分得到的每个三角形,对源图像中的对应三角形区域进行仿射变换,将其扭曲到目标图像中相应三角形的位置。
	然后,使用OpenCV的 \texttt{cv2.seamlessClone} 函数,将扭曲后的面部区域无缝地融合到目标图像中,完成换脸。
	\texttt{seamlessClone} 通过泊松图像编辑技术,确保融合区域的梯度平滑过渡,避免明显的接缝。
\end{enumerate}

\subsection{试用大模型demo换脸技术}
\begin{enumerate}
	\item 登录网站https://modelscope.cn/studios/Hardwell/hardwell_face_fusion_light_release;
	\item 上传想要的两张图片并保存输出图像
\end{enumerate}
